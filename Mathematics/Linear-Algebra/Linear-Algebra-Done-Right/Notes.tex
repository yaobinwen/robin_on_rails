% Regarding `oneside` (https://stackoverflow.com/a/8371473/630364):
%
% `oneside` removes the blank pages between chapters.
% "Note that this method make the margins of all the pages the same. In
% `twoside`, the margins are different for the odd and the even pages".
\documentclass[12pt, letterpaper, oneside]{book}
\usepackage{amsmath}
\usepackage[letterpaper, textwidth=7.5in, textheight=8in]{geometry}
\usepackage{csquotes}

\title{Notes on \textit{Linear Algebra Done Right}}
\author{Yaobin Wen}
\date{January 2023}

\begin{document}

\maketitle
\tableofcontents

\chapter*{Overview}
\addcontentsline{toc}{chapter}{Preface}

This document contains my study notes of the textbook \textit{Linear Algebra
Done Right}. I use it for a few purposes:

\begin{enumerate}
  \item As a reference to quickly refresh my memory on the subjects.
  \item Keep the notes to help me understand the text that is not obvious for
    me to comprehend.
  \item Keep the notes of solving the exercises.
\end{enumerate}

% =============================================================================
%
% Chapter 1: Vector Spaces
%
% =============================================================================

\chapter{Vector Spaces}

\emph{Main topics of this chapter}
\begin{itemize}
  \item Complex numbers and properties
  \item List; length
  \item $\mathbf{F}^n$ and properties
  \item Scalars; vectors
\end{itemize}

\section*{1.1 Definition: complex numbers}

\begin{itemize}
  \item A \textbf{\emph{complex number}} is an ordered pair $(a, b)$, where $a,
    b \in \mathbf{R}$, but we will write this as $a + bi$.
  \begin{itemize}
    \item When $a, b \in \mathbf{R}$ and $b = 0$, we have all the real numbers
      $\mathbf{R}$. Thus $\mathbf{R} \subset \mathbf{C}$.
  \end{itemize}
  \item The set of all complex numbers is denoted by $\mathbf{C}$:
    \[ \mathbf{C} = \{a + bi: a, b \in \mathbf{R}\} \]
  \item \textbf{Addition}: For $a, b, c, d \in \mathbf{R}$:
    \[ (a + bi) + (c + di) = (a + c) + (b + d)i \]
  \item \textbf{Multiplication}: For $a, b, c, d \in \mathbf{R}$:
    \[ (a + bi)(c + di) = (ac - bd) + (ad + bc)i \]
\end{itemize}

\section*{1.3 Properties of complex arithmetic}

\begin{itemize}
  \item \textbf{commutativity}:
    \[
      \alpha + \beta = \beta + \alpha \quad and \quad \alpha\beta\ =
      \beta\alpha \quad for \; all \; \alpha, \beta \in \mathbf{C}
    \]
  \item \textbf{associativity}:
    \[
      (\alpha + \beta) + \lambda = \alpha + (\beta + \lambda) \quad and \quad
      (\alpha\beta)\lambda = \alpha(\beta\lambda) \quad for \; all \; \alpha,
      \beta, \lambda \in \mathbf{C}
    \]
  \item \textbf{identities}:
    \[
      \lambda + 0 = \lambda \ and \ \lambda1 = \lambda \ for \ all \ \lambda
      \in \mathbf{C}
    \]
  \item \textbf{additive inverse}:
    \[
      For \ every \ \alpha \in \mathbf{C}, there \ exists \ a \ unique \ \beta
      \in \mathbf{C} \ such \ that \ \alpha + \beta = 0
    \]
  \item \textbf{multiplicative inverse}: \newline
     $For \ every \ \alpha \in \mathbf{C} \ with \ \alpha \neq 0, there \
     exists \ a \ unique \ \beta \in \mathbf{C} \newline such \ that \
     \alpha\beta = 1$
  \item \textbf{distributive property}:
    \[
      \lambda(\alpha + \beta) = \lambda\alpha + \lambda\beta \ for \ all \
      \lambda, \alpha, \beta \in \mathbf{C}
    \]
\end{itemize}

NOTE(ywen): Complex numbers are defined based upon real numbers. In order to
prove the properties of complex arithmetic, we need to use the properties of
real arithmetic (i.e., communtativity, associativity, etc. for real numbers).
Example 1.4 shows this method.

\section*{1.5 Definition: $-\alpha$, subtraction, $1/\alpha$, division}

Let $\alpha, \beta \in \mathbf{C}$:

\begin{itemize}
  \item ``$-\alpha$'' is the additive inverse of $\alpha$ such that
    \[ \alpha + (-\alpha) = 0 \]
  \item \textbf{\textit{Subtraction}} on $\mathbf{C}$ is defined by
    \[ \beta - \alpha = \beta + (-\alpha) \]
  \item For $\alpha \neq 0$, define $1/\alpha$ as the \textbf{multiplicative
    inverse} of $\alpha$ such that
    \[ \alpha(1/\alpha) = 1 \]
  \item \textbf{\textit{Division}} on $\mathbf{C}$ is defined by
    \[ \beta/\alpha = \beta(1/\alpha) \]
\end{itemize}

\section*{Definition: $\alpha^m$}

For $\alpha \in \mathbf{F}$ and $m \in \mathbf{N}$:

\[ \alpha^m = \underbrace{\alpha\cdot\cdot\cdot\alpha}_{m \ times} \]

Thus for all $\alpha, \beta \in \mathbf{F}$, $m, n \in \mathbf{N}$:

\begin{itemize}
  \item $(\alpha^m)^n = \alpha^{mn}$
  \item $(\alpha\beta)^m = \alpha^m\beta^m$
\end{itemize}

\section*{1.8 Definition: \textit{list}, \textit{length}}

For $n \in \mathbf{N}$, a \textbf{\textit{list}} of \textbf{\textit{length n}}
is an \textbf{ordered} collection of \textit{n} \textbf{elements}. It's denoted
as follows: \[ (x_1, x_2, ..., x_n) \]

Note that:
\begin{itemize}
  \item A list has a \textbf{finite} length.
  \item An element can be anything: a number, another list, a text string, etc.
  \item A list of length 0 is denoted as ``$()$''.
  \item A list of length n is also called an \textit{n-\textbf{tuple}}.
\end{itemize}

Two lists $(x_1, x_2, ..., x_n)$ and $(y_1, y_2, ..., y_m)$ are equal if and
only if:
\begin{itemize}
  \item $n = m$
  \item $x_i = y_i$ for $i = 1, 2, ..., n$
\end{itemize}

\section*{Lists and Sets}

Table \ref{table:lists_sets_comp} compares lists and sets:
\begin{table}[h!]
\centering
\begin{tabular}{||c c c ||} 
 \hline
   & Lists & Sets \\ [0.5ex] 
 \hline
 \hline
 Length & Finite & Finite or infinite \\ 
 Order & Matters & Doesn't matter \\
 Repetition & Allows & Doesn't allow \\ [1ex]
 \hline
\end{tabular}
\caption{Compare lists and sets}
\label{table:lists_sets_comp}
\end{table}

\section*{1.10 Definition: $\mathbf{F^n}$ and $0$}

We use $\mathbf{F}$ to denote either $\mathbf{R}$ or $\mathbf{C}$.

We define $\mathbf{F^n}$ as the set of all lists of length $n$ of elements of
$\mathbf{F}$:

\[
  \mathbf{F^n} = \{(x_1, ..., x_n): x_j \in \mathbf{F} \ for \ j = 1, ..., n\}
\]

We say $x_j$ is the $j^{th}$ \textbf{\textit{coordinate}} of $(x_1, ..., x_n)$.

Let $\mathbf{0}$ denote the list of length $n$ whose coordinates are all $0$:

\[
  \mathbf{0} = (0, 0, ..., 0)
\]

\section*{1.11 Note: $\mathbf{C^1}$ can be thought of as a plane}

On page 6, below Example 1.11, the textbook says:

\begin{displayquote}
  Similarly, $\mathbf{C}^1$ can be thought of as a plane.
\end{displayquote}

It is so because $\mathbf{C}^1$ is the set of complex numbers, each of which
consists of two parts: the real part and the imaginary part. Both parts are real
numbers and they can form a 2D plane, hence ``as a plane''.

\section*{1.12 Vectors}

For $x \in \mathbf{F}^n$, we can view it as a point in the space $\mathbf{F}^n$,
but we can also view it as an ``arrow'' in the same space. When we view it as
an ``arrow'', we refer to it as a \textbf{\textit{vector}}.

\section*{1.13 \textit{addition} in $\mathbf{F}^n$}

Given

\[
  \mathbf{F^n} = \{(x_1, ..., x_n): x_j \in \mathbf{F} \ for \ j = 1, ..., n\}
\]

and two lists $x, y \in \mathbf{F}^n$, and

\[
  x = (x_1, x_2, ..., x_n)
\]
\[
  y = (y_1, y_2, ..., y_n)
\]

we have

\begin{equation*}
\begin{split}
  x + y &= (x_1, x_2, ..., x_n) + (y_1, y_2, ..., y_n) \\
        &= (x_1 + y_1, x_2 + y_2, ..., x_n + y_n)
\end{split}
\end{equation*}

The addition has the property of \textbf{\textit{commutativity}}: If $x, y \in
\mathbf{F}^n$, then $x + y = y + x$.

The \textbf{\textit{additive inverse}} of $x$, denoted as $-x$, is the vector
$-x \in \mathbf{F}^n$ such that

\[
  x + (-x) = 0
\]

In other words, if $x = (x_1, ..., x_n)$, then $-x = (-x_1, ..., -x_n)$.

\section*{1.14 Definition:\
  \textbf{\textit{scalar multiplication}} in $\mathbf{F}^n$}

For $\lambda \in \mathbf{F}$ and $(x_1, ..., x_n) \in \mathbf{F}^n$, we define

\[
  \lambda(x_1, ..., x_n) = (\lambda x_1, ..., \lambda x_n)
\]

\section*{TODOs}

\begin{itemize}
  \item Define ``scalar''.
\end{itemize}

\section*{Progress tracker}

As of 2023-02-21, I'm on page 10: Digression on Fields

\end{document}
