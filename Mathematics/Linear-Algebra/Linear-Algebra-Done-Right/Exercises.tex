\documentclass[12pt, letterpaper, oneside]{book}
\usepackage{amsmath}
\usepackage{amssymb}
\usepackage{empheq}
\usepackage[letterpaper, textwidth=7.5in, textheight=8in]{geometry}
\usepackage{csquotes}
\usepackage{parskip}

\title{Exercises in \textit{Linear Algebra Done Right}}
\author{Yaobin Wen}
\date{February 2023}

\begin{document}

\maketitle
\tableofcontents

% =============================================================================
%
% Chapter 1: Vector Spaces
%
% =============================================================================

\chapter{Vector Spaces}

% =============================================================================
\section{EXERCISES 1.A}
% =============================================================================

% ******************************
\subsection{Question 1}
% ******************************

Description: Suppose $a$ and $b$ are real numbers, not both 0. Find real numbers $c$ and $d$
such that

\[
  \frac{1}{(a+bi)} = c + di
\]

Answer:

$\frac{1}{(a+bi)} = c + di \Rightarrow (a+bi)(c+di) = 1 \Rightarrow (ac-bd) + (ad + bc)i = 1$

$\therefore$ We have:

\begin{empheq}[left=\empheqlbrace]{align}
  &ac - bd = 1 \\
  &ad + bc = 0
\end{empheq}

Because $a$ and $b$ are not both 0, there can be three cases.

Case 1: $a = 0$, $b \neq 0$

\[(1.1) \ and \ (1.2)\]

$\Rightarrow$

\begin{empheq}[left=\empheqlbrace]{align*}
  -bd &= 1 \\
  bc &= 0
\end{empheq}

$\Rightarrow$

\begin{empheq}[left=\empheqlbrace]{align*}
  &c = 0 \\
  &d = -\frac{1}{b}
\end{empheq}

Case 2: $a \neq 0$, $b = 0$

\[(1.1) \ and \ (1.2)\]

$\Rightarrow$

\begin{empheq}[left=\empheqlbrace]{align*}
  &ac = 1 \\
  &ad = 0
\end{empheq}

$\Rightarrow$

\begin{empheq}[left=\empheqlbrace]{align*}
  &c = \frac{1}{a} \\
  &d = 0
\end{empheq}

Case 3: $a \neq 0$, $b \neq 0$

\[(1.1) \ and \ (1.2)\]

$\Rightarrow$

\begin{empheq}[left=\empheqlbrace]{align*}
  &c = \frac{a}{a^2+b^2} \\
  &d = -\frac{b}{a^2+b^2}
\end{empheq}

(Done)

% ******************************
\subsection{Question 2}
% ******************************

Description: Show that \[ \frac{-1 + \sqrt{3}i}{2} \] is a cube root of $1$
(meaning that its cube equals $1$).

Solution:

\begin{equation*}
  \begin{split}
    (\frac{-1 + \sqrt{3}i}{2})^3
    & = (\frac{-1 + \sqrt{3}i}{2})^2 \cdot \frac{-1 + \sqrt{3}i}{2} \\
    & = \frac{1}{4}\bigl[ (-1)^2 - \sqrt{3}i - \sqrt{3}i + 3i^2 \bigr] \cdot
      \frac{-1 + \sqrt{3}i}{2} \\
    & = \frac{1}{4}\bigl[ 1 - 2\sqrt{3}i - 3 \bigr] \cdot
      \frac{-1 + \sqrt{3}i}{2} \\
    & = -\frac{1 + \sqrt{3}i}{2} \cdot \frac{-1 + \sqrt{3}i}{2} \\
    & = -\frac{1}{4}\bigl(-1 + \sqrt{3}i - \sqrt{3}i + 3i^2 \bigr) \\
    & = -\frac{1}{4} \cdot (-1 - 3) \\
    & = -\frac{1}{4} \cdot (-4) \\
    & = 1
  \end{split}
\end{equation*}

$\therefore$: $\frac{-1 + \sqrt{3}i}{2}$ is a cube root of $1$.

% ******************************
\subsection{Question 3}
% ******************************

TODO

% ******************************
\subsection{Question 4}
% ******************************

TODO

% ******************************
\subsection{Question 5}
% ******************************

TODO

% ******************************
\subsection{Question 6}
% ******************************

TODO

% ******************************
\subsection{Question 7}
% ******************************

TODO

% ******************************
\subsection{Question 8}
% ******************************

TODO

% ******************************
\subsection{Question 9}
% ******************************

TODO

% ******************************
\subsection{Question 10}
% ******************************

TODO

% ******************************
\subsection{Question 11}
% ******************************

Explain why there does not exist $\lambda \in \mathbf{C}$ such that
\[\lambda(2-3i, 5+4i, -6+7i) = (12-5i, 7+22i, -32-9i) \]

Solution:

According to the definition of $\mathbf{F}^n$, $(2-3i, 5+4i, -6+7i)$ and
$(12-5i, 7+22i, -32-9i)$ are two lists when $n=3$.

Because $\lambda \in \mathbf{C}$ and $\mathbf{F}$ can denote $\mathbf{C}$,
we have $\lambda \in \mathbf{F}$.

Therefore, the expression $\lambda(2-3i, 5+4i, -6+7i)$ is a \textit{scalar
multiplication} on $\mathbf{F}^3$, so we have:

\[
  \lambda(2-3i, 5+4i, -6+7i) = (\lambda(2-3i), \lambda(5+4i), \lambda(-6+7i))
\]

\begin{equation*}
  \begin{split}
    \lambda(2-3i, 5+4i, -6+7i)
    & = (\lambda(2-3i), \lambda(5+4i), \lambda(-6+7i)) \\
    & = (2\lambda - 3\lambda i, 5\lambda + 4\lambda i, -6\lambda + 7\lambda i)
  \end{split}
\end{equation*}

To make it equal to $(12-5i, 7+22i, -32-9i)$, we must have:

\begin{empheq}[left=\empheqlbrace]{align*}
  2\lambda - 3\lambda i & = 12-5i \\
  5\lambda + 4\lambda i & = 7+22i \\
  -6\lambda + 7\lambda i & = -32-9i
\end{empheq}

This can be further reduced as:

\begin{empheq}[left=\empheqlbrace]{align*}
  2\lambda & = 12 \\
  -3\lambda &= -5 \\
  5\lambda & = 7 \\
  4\lambda & = 22 \\
  -6\lambda & = -32 \\
  7\lambda & = -9
\end{empheq}

Unfortunately, we can't find a single $\lambda$ that satisfies all of the
equations above. Therefore, such a $\lambda$ doesn't exist.

% ******************************
\subsection{Question 12}
% ******************************

Show that $(x + y) + z = x + (y + z)$ for all $x, y, z \in \mathbf{F}^n$.

Solution:

Because $x, y, z \in \mathbf{F}^n$, according to the definition of
$\mathbf{F}^n$, we know that:

\[ x = (x_1, \ldots, x_n), x_j \in \mathbf{F}, j = 1, \ldots, n \]
\[ y = (y_1, \ldots, y_n), y_j \in \mathbf{F}, j = 1, \ldots, n \]
\[ z = (z_1, \ldots, z_n), z_j \in \mathbf{F}, j = 1, \ldots, n \]

Because $x_j, y_j, z_j \in \mathbf{F}$, their addition satisfies the property
of associativity, i.e.,

\[ (a + b) + c = a + (b + c), a, b, c \in \mathbf{F} \]

Therefore, according to the definition of addition on $\mathbf{F}^n$, we have:

\begin{equation*}
  \begin{split}
    (x + y) + z
    & = \bigl((x_1, \ldots, x_n) + (y_1, \ldots, y_n)\bigr) +
      (z_1, \ldots, z_n) \\
    & = (x_1 + y_1, \ldots, x_n + y_n) + (z_1, \ldots, z_n) \\
    & = \bigl((x_1 + y_1), \ldots, (x_n + y_n)\bigr) + (z_1, \ldots, z_n) \\
    & = \bigl((x_1 + y_1) + z_1, \ldots, (x_n + y_n) + z_n\bigr) \\
    & = \bigl(x_1 + (y_1 + z_1), \ldots, x_n + (y_n + z_n)\bigr) \\
    & = (x_1, \ldots, x_n) + \bigl((y_1 + z_1), \ldots, (y_n + z_n)\bigr) \\
    & = (x_1, \ldots, x_n) + \bigl(y_1 + z_1, \ldots, y_n + z_n\bigr) \\
    & = (x_1, \ldots, x_n) + \bigl((y_1, \ldots, y_n) +
      (z_1, \ldots, z_n)\bigr) \\
    & = x + (y + z)
  \end{split}
\end{equation*}

% ******************************
\subsection{Question 13}
% ******************************

Show that $(ab)x = a(bx)$ for all $x \in \mathbf{F}^n$ and all $a, b \in
\mathbf{F}$

Solution:

Because $x \in \mathbf{F}^n$, according to the definition of $\mathbf{F}^n$, we
know that:

\[ x = (x_1, \ldots, x_n), x_j \in \mathbf{F}, j = 1, \ldots, n \]

Because $a, b, x_j \in \mathbf{F}$, their multiplication satisfies the property
of associativity, i.e.,

\[ (ab)x_j = a(b x_j), a, b, x_j \in \mathbf{F} \]

Because $a, b \in \mathbf{F}$, we have $ab \in \mathbf{F}$. Therefore, $(ab)x$
is the case of scalar multiplication on $\mathbf{F}^n$. Therefore, according to
the definition of scalar multiplication on $\mathbf{F}^n$, we have:

\begin{equation*}
  \begin{split}
    (ab)x
    & = (ab)(x_1, \ldots, x_n) \\
    & = \bigl((ab)x_1, \ldots, (ab)x_n\bigr) \\
    & = \bigl(a(b x_1), \ldots, a(b x_n)\bigr) \\
    & = a(b x_1, \ldots, b x_n) \\
    & = a\bigl(b(x_1, \ldots, x_n)\bigr) \\
    & = a(bx)
  \end{split}
\end{equation*}

% ******************************
\subsection{Question 14}
% ******************************

Show that $1x = x$ for all $x \in \mathbf{F}^n$.

Solution:

Because we have never defined the multiplication of two elements in
$\mathbf{F}^n$ throughout the entire chapter, and because so far we have only
defined scalar multiplication on $\mathbf{F}^n$, we have $1$ must be a scalar
in $\mathbf{F}$.

Because $x \in \mathbf{F}^n$, according to the definition of $\mathbf{F}^n$, we
know that:

\[ x = (x_1, \ldots, x_n), x_j \in \mathbf{F}, j = 1, \ldots, n \]

Because $1$ is the multiplicative identity in $\mathbf{F}$, we know that

\[ 1a = a, \forall a \in \mathbf{F} \]

Therefore, according to the definition of scalar multiplication on
$\mathbf{F}^n$, we have:

\begin{equation*}
  \begin{split}
    1x
    & = 1(x_1, \ldots, x_n) \\
    & = (1 x_1, \ldots, 1 x_n) \\
    & = (x_1, \ldots, x_n) \\
    & = x
  \end{split}
\end{equation*}

% ******************************
\subsection{Question 15}
% ******************************

Show that $\lambda(x + y) = \lambda x + \lambda y$ for all $\lambda \in
\mathbf{F}$ and all $x, y \in \mathbf{F}^n$.

Solution:

Because $x, y \in \mathbf{F}^n$, according to the definition of $\mathbf{F}^n$,
we know that:

\[ x = (x_1, \ldots, x_n), x_j \in \mathbf{F}, j = 1, \ldots, n \]
\[ y = (y_1, \ldots, y_n), y_j \in \mathbf{F}, j = 1, \ldots, n \]

Because $\lambda, x_j, y_j \in \mathbf{F}$, the distributive property holds:

\[ \lambda(x_j + y_j) = \lambda x_j + \lambda y_j \]

Therefore, according to the addition and scalar multiplication on
$\mathbf{F}^n$, we have:

\begin{equation*}
  \begin{split}
    \lambda(x + y)
    & = \lambda \bigl((x_1, \ldots, x_n) + (y_1, \ldots, y_n)\bigr) \\
    & = \lambda (x_1 + y_1, \ldots, x_n + y_n) \\
    & = \bigl(\lambda(x_1 + y_1), \ldots, \lambda(x_n + y_n)\bigr) \\
    & = \bigl(\lambda x_1 + \lambda y_1, \ldots,
      \lambda x_n + \lambda y_n\bigr) \\
    & = (\lambda x_1, \ldots, \lambda x_n) +
      (\lambda y_1, \ldots, \lambda y_n) \\
    & = \lambda x + \lambda y
  \end{split}
\end{equation*}

% ******************************
\subsection{Question 16}
% ******************************

Show that $(a + b)x = ax + bx$ for all $a, b \in \mathbf{F}$ and all $x \in
\mathbf{F}^n$.

Solution:

Because $x \in \mathbf{F}^n$, according to the definition of $\mathbf{F}^n$, we
know that:

\[ x = (x_1, \ldots, x_n), x_j \in \mathbf{F}, j = 1, \ldots, n \]

Because $a, b, x_j \in \mathbf{F}$, their arithmetics satisfy the commutativity
and distributive property:

\begin{equation*}
  \begin{split}
    (a + b)x_j
    & = x_j (a + b) \\
    & = x_j a + x_j b \\
    & = a x_j + b x_j
  \end{split}
\end{equation*}

Therefore, according to the scalar multiplication on $\mathbf{F}^n$, we have:

\begin{equation*}
  \begin{split}
    (a + b)x
    & = (a + b)(x_1, \ldots, x_n) \\
    & = \bigl((a + b)x_1, \ldots, (a + b)x_n\bigr) \\
    & = (a x_1 + b x_1, \ldots, a x_n + b x_n) \\
    & = (a x_1, \ldots, a x_n) + (b x_1, \ldots, b x_n) \\
    & = ax + bx
  \end{split}
\end{equation*}

\end{document}
